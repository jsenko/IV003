%%%%%%%%%%%%%%%%%%%%%%%%%%%%%%%%%%%%%%%%%
% Short Sectioned Assignment
% LaTeX Template
% Version 1.0 (5/5/12)
%
% This template has been downloaded from:
% http://www.LaTeXTemplates.com
%
% Original author:
% Frits Wenneker (http://www.howtotex.com)
%
% License:
% CC BY-NC-SA 3.0 (http://creativecommons.org/licenses/by-nc-sa/3.0/)
%
%%%%%%%%%%%%%%%%%%%%%%%%%%%%%%%%%%%%%%%%%

%----------------------------------------------------------------------------------------
%	PACKAGES AND OTHER DOCUMENT CONFIGURATIONS
%----------------------------------------------------------------------------------------

\documentclass[paper=a4, fontsize=11pt]{scrartcl} % A4 paper and 11pt font size

\usepackage[T1]{fontenc} % Use 8-bit encoding that has 256 glyphs
% \usepackage{fourier} % Use the Adobe Utopia font for the document - comment this line to return to the LaTeX default
\usepackage[slovak]{babel} % Slovak language/hyphenation
\usepackage[utf8]{inputenc}
% \usepackage[IL2]{fontenc}
\usepackage{amsmath,amsfonts,amsthm} % Math packages
\usepackage{algpseudocode}
\usepackage{lipsum} % Used for inserting dummy 'Lorem ipsum' text into the template

\usepackage{sectsty} % Allows customizing section commands
\allsectionsfont{\centering \normalfont\scshape} % Make all sections centered, the default font and small caps

\usepackage{fancyhdr} % Custom headers and footers
\pagestyle{fancyplain} % Makes all pages in the document conform to the custom headers and footers
\fancyhead{} % No page header - if you want one, create it in the same way as the footers below
\fancyfoot[L]{} % Empty left footer
\fancyfoot[C]{} % Empty center footer
\fancyfoot[R]{\thepage} % Page numbering for right footer
\renewcommand{\headrulewidth}{0pt} % Remove header underlines
\renewcommand{\footrulewidth}{0pt} % Remove footer underlines
\setlength{\headheight}{13.6pt} % Customize the height of the header

\numberwithin{equation}{section} % Number equations within sections (i.e. 1.1, 1.2, 2.1, 2.2 instead of 1, 2, 3, 4)
\numberwithin{figure}{section} % Number figures within sections (i.e. 1.1, 1.2, 2.1, 2.2 instead of 1, 2, 3, 4)
\numberwithin{table}{section} % Number tables within sections (i.e. 1.1, 1.2, 2.1, 2.2 instead of 1, 2, 3, 4)

\setlength\parindent{0pt} % Removes all indentation from paragraphs - comment this line for an assignment with lots of text

%----------------------------------------------------------------------------------------
%	TITLE SECTION
%----------------------------------------------------------------------------------------

\newcommand{\horrule}[1]{\rule{\linewidth}{#1}} % Create horizontal rule command with 1 argument of height

\title{	
\normalfont \normalsize 
\textsc{IV003, Fakulta Informatiky, Masarykova Univerzita} \\ [25pt] % Your university, school and/or department name(s)
\horrule{0.5pt} \\[0.4cm] % Thin top horizontal rule
\huge Homework 1 \\ % The assignment title
\horrule{2pt} \\[0.5cm] % Thick bottom horizontal rule
}

\author{Jakub Senko, Štefan Uherčík} % Your name

\date{\normalsize\today} % Today's date or a custom date

\begin{document}

\maketitle % Print the title

%----------------------------------------------------------------------------------------
%	PROBLEM 1
%----------------------------------------------------------------------------------------

\section*{Problem 1}

Nech $X = [x_1, x_2, x_3, \dots, x_n]$ je pole cisel a plati ze $\forall x,y \in X: x \neq y$. \\
Kazdemu $x_i \in X$ je priradene cislo $w_i$, pre ktore plati:
\begin{equation}
    \begin{aligned}
        w_i > 0 \\
        \sum_{i = 0}^{n} w_i = 1
    \end{aligned}
\end{equation}
{\em Optimalny prvok} postupnosti je cislo $x_k$ pre ktore plati:

\begin{equation}
    \begin{aligned}
        \sum_{x_i < x_k} w_i < \frac{1}{2} \\
        \sum_{x_i > x_k} w_i \leq \frac{1}{2}
    \end{aligned}
\end{equation}

Problemom je navrh algoritmu ktory riesi najdenie optimalneho prvku s casovou zlozitostou $\Theta(n)$ a poskytnutie dokazu jeho korektnosti a zlozitosti. \\
\\
Navrhovane riesenie je modifikovany algoritmus {\em Quick Select} ktory riesi problem najdenia medianu v poli cisel. Tento algoritmus ma obecne zlozitost $\mathcal{O}(n^2)$ pri nevhodnej volbe pivota, avsak pomocou procedury {\em Median of Medians} je mozne najst dostatocne dobry pivot na to, aby mal algoritmus vzdy linearnu zlozitost.  {\em Quick Select} je popisany v nasledujucom texte iba neformalne, s odkazom na relevantne zdroje s dokazom zlozitosti. Zadana uloha  je vyriesena ukazanim redukcie problemu najdenia optimalneho prvku na problem rieseny algoritmom {\em Quick Select + Median of Medians} a dokazom ze tato procedura je vykonatelna v konstantom case. Vysledna zlozitost je teda $\mathcal{O}(n)$. \\
\\
Quick Select \\

\begin{algorithmic}[5]
    \While{true}
        \State $user \gets beer$
    \EndWhile
\end{algorithmic}


%----------------------------------------------------------------------------------------
%	PROBLEM 2
%----------------------------------------------------------------------------------------

\section*{Problem 2}

\subsection{Example of list (3*itemize)}
\begin{itemize}
	\item First item in a list 
		\begin{itemize}
		\item First item in a list 
			\begin{itemize}
			\item First item in a list 
			\item Second item in a list 
			\end{itemize}
		\item Second item in a list 
		\end{itemize}
	\item Second item in a list 
\end{itemize}

%------------------------------------------------

\subsection{Example of list (enumerate)}
\begin{enumerate}
\item First item in a list 
\item Second item in a list 
\item Third item in a list
\end{enumerate}

%----------------------------------------------------------------------------------------

\end{document}