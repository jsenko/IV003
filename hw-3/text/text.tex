%%%%%%%%%%%%%%%%%%%%%%%%%%%%%%%%%%%%%%%%%
% Short Sectioned Assignment
% LaTeX Template
% Version 1.0 (5/5/12)
%
% This template has been downloaded from:
% http://www.LaTeXTemplates.com
%
% Original author:
% Frits Wenneker (http://www.howtotex.com)
%
% License:
% CC BY-NC-SA 3.0 (http://creativecommons.org/licenses/by-nc-sa/3.0/)
%
%%%%%%%%%%%%%%%%%%%%%%%%%%%%%%%%%%%%%%%%%

%----------------------------------------------------------------------------------------
%	PACKAGES AND OTHER DOCUMENT CONFIGURATIONS
%----------------------------------------------------------------------------------------

\documentclass[paper=a4, fontsize=11pt]{scrartcl} % A4 paper and 11pt font size

\usepackage[T1]{fontenc} % Use 8-bit encoding that has 256 glyphs
% \usepackage{fourier} % Use the Adobe Utopia font for the document - comment this line to return to the LaTeX default
\usepackage[slovak]{babel} % Slovak language/hyphenation
\usepackage[utf8]{inputenc}
% \usepackage[IL2]{fontenc}
\usepackage{amsmath,amsfonts,amsthm} % Math packages
\usepackage{algpseudocode}
\usepackage{lipsum} % Used for inserting dummy 'Lorem ipsum' text into the template

\usepackage{sectsty} % Allows customizing section commands
\allsectionsfont{\centering \normalfont\scshape} % Make all sections centered, the default font and small caps

\usepackage{fancyhdr} % Custom headers and footers
\pagestyle{fancyplain} % Makes all pages in the document conform to the custom headers and footers
\fancyhead{} % No page header - if you want one, create it in the same way as the footers below
\fancyfoot[L]{} % Empty left footer
\fancyfoot[C]{} % Empty center footer
\fancyfoot[R]{\thepage} % Page numbering for right footer
\renewcommand{\headrulewidth}{0pt} % Remove header underlines
\renewcommand{\footrulewidth}{0pt} % Remove footer underlines
\setlength{\headheight}{13.6pt} % Customize the height of the header

\numberwithin{equation}{section} % Number equations within sections (i.e. 1.1, 1.2, 2.1, 2.2 instead of 1, 2, 3, 4)
\numberwithin{figure}{section} % Number figures within sections (i.e. 1.1, 1.2, 2.1, 2.2 instead of 1, 2, 3, 4)
\numberwithin{table}{section} % Number tables within sections (i.e. 1.1, 1.2, 2.1, 2.2 instead of 1, 2, 3, 4)

\setlength\parindent{0pt} % Removes all indentation from paragraphs - comment this line for an assignment with lots of text

%----------------------------------------------------------------------------------------
%	TITLE SECTION
%----------------------------------------------------------------------------------------

\newcommand{\horrule}[1]{\rule{\linewidth}{#1}} % Create horizontal rule command with 1 argument of height

\title{	
\normalfont \normalsize 
\textsc{IV003, Fakulta Informatiky, Masarykova Univerzita} \\ [25pt] % Your university, school and/or department name(s)
\horrule{0.5pt} \\[0.4cm] % Thin top horizontal rule
\huge Homework 3 \\ % The assignment title
\horrule{2pt} \\[0.5cm] % Thick bottom horizontal rule
}

\author{Jakub Senko, Štefan Uherčík} % Your name

\date{\normalsize\today} % Today's date or a custom date

\begin{document}

%\maketitle  Print the title

%----------------------------------------------------------------------------------------
%	Príklad 1
%----------------------------------------------------------------------------------------

\section*{IV003 2014, Sada 3, Príklad 1}
\subsection*{Jakub Senko (373902), Štefan Uherčík (374375)}

\pagebreak

%----------------------------------------------------------------------------------------
%	Príklad 2
%----------------------------------------------------------------------------------------

\section*{IV003 2014, Sada 3, Príklad 2}
\subsection*{Jakub Senko (373902), Štefan Uherčík (374375)}

\pagebreak

%----------------------------------------------------------------------------------------
%	Príklad 3
%----------------------------------------------------------------------------------------

\section*{IV003 2014, Sada 3, Príklad 3}
\subsection*{Jakub Senko (373902), Štefan Uherčík (374375)}

V nasledovných algoritmoch je graf reprezentovaný maticou susednosti, kde $M[i][j]$ je ohodnotenie hrany $i \to j$. Neexistujúcu hranu ohodnotíme znakom $\perp$. Zadaný graf označme $G$. Predpokladáme že $G$ neobsahuje hrany $u \to u$.

\subsection*{Reduce}

Navrhneme funkciu REDUCE, ktorá odstráni vrchol $node$ tak, aby sa nezmenilo ohodnotenie {\em najkratších ciest} medzi ostatnými vrcholmi.\\
\ \\
Pre každú najkratšiu cestu $u \leadsto i \to node \to j \leadsto v$, je potrebné nahradiť podcestu $i \to node \to j$ novou hranou $i \to j$ tak, aby $G[i][j] = G[i][node] + G[node][j]$. Ak hrana $i \to j$ už existuje, stačí ponechať hranu s nižším ohodnotením.\\
\ \\
Nahradenie je potrebné vykonať pre všetky možné podcesty, algoritmus teda obsahuje dva vnorené cykly. Vonkajší cyklus iteruje cez vstupné hrany $i \to node$ a vnútorný cez výstupné hrany $node \to j$. Po skončení REDUCE na vrchole $node$ sú všetky ohodnotenia jeho hrán nastavené na $\perp$ čo reprezentuje ich zrušenie.\\
\ \\
\begin{algorithmic}[1]
    \Function{REDUCE}{$G[1 \dots n][1 \dots n], node$}
        \For{$i \gets 1 \dots n$}
            \If{$G[i][node] \neq \perp \wedge \ i \neq node$}
                \For{$j \gets 1 \dots n$}
                   \If{$G[node][j] \neq \perp \wedge \ j \neq node$}
                       \State{$cost \gets G[i][node] + G[node][j]$}
                       \If{$G[i][j] = \perp \vee \ G[i][j] > cost$}
                           \State{$G[i][j] \gets cost$}
                       \EndIf
                   \EndIf
                \EndFor                
                \State{$G[i][node] \gets \perp$}
            \EndIf            
        \EndFor
       \For{$j \gets 1 \dots n$}
           \State{$G[node][j] \gets \perp$}
       \EndFor
        \State \Return{$G$}
    \EndFunction
\end{algorithmic}
\ \\
Zlozitost je určená dvoma cyklami - $\mathcal{O}(n^2)$.

\subsection*{AUGMENT}

Predpokladáme že matica najkratších ciest $\mathcal{G}$ pre podmnožinu vrcholov $G$ má rovnaké rozmery ako $G$.  Vstupom nasledovného algoritmu je $G$, $\mathcal{G}$ a vrchol $node \in V_{G} \setminus V_{\mathcal{G}}$. Výstupom je matica najkratších ciest obsahujúca $node$.\\
\ \\
Algoritmus je rozdelený na dve funkcie, zvlášť pre vstupné a pre výstupné hrany. Funkcie sú podobné, pre krátkosť teda popíšeme len AUGMENT\_OUT pre výstupné hrany:\\
\ \\
Nech $u \to v$ je hrana v $\mathcal{G}$. Pre každú hranu $node \to u$ v grafe $G$ pridáme do $\mathcal{G}$ hranu $node \to v$ s ohodnotením $G[node][u] +\mathcal{G}[u][v]$. Kedže vrchol $u$ nie je unikátny, táto hrana už môže počas priebehu funkcie existovať. Znovu použijeme nižšie ohodnotenie.\\
\ \\
\begin{algorithmic}[1]
    \Function{AUGMENT\_OUT}{$G[1 \dots n][1 \dots n], \mathcal{G}[1 \dots n][1 \dots n], node$}
        \State{$\mathcal{G}[node][node] \gets 0$}
        \For{$u \gets 1 \dots n$}
            \If{$G[node][u] \neq \perp \wedge \ u \neq node$}
                \For{$v \gets 1 \dots n$}
                    \If{$\mathcal{G}[u][v] \neq \perp \wedge \ v \neq node$}
                        \State{$cost \gets G[node][u] + \mathcal{G}[u][v]$}
                        \If{$\mathcal{G}[node][v] = \perp \vee \ \mathcal{G}[node][v] > cost$}
                            \State{$\mathcal{G}[node][v] \gets cost$}
                        \EndIf
                    \EndIf
                \EndFor
            \EndIf
        \EndFor
        \State \Return{$\mathcal{G}$}
    \EndFunction
\end{algorithmic}
\ \\
Výsledná funkcia AUGMENT je nasledovná:\\
\ \\
\begin{algorithmic}[1]
    \Function{AUGMENT}{$G[1 \dots n][1 \dots n], \mathcal{G}[1 \dots n][1 \dots n], node$}
        \State{$\mathcal{G} \gets AUGMENT\_IN(G, \mathcal{G}, node)$}
        \State{$\mathcal{G} \gets AUGMENT\_OUT(G, \mathcal{G}, node)$}
        \State \Return{$\mathcal{G}$}
    \EndFunction
\end{algorithmic}
\ \\
Zložitosť je určená dvoma vnorenými cyklami vo funkcii AUGMENT\_OUT/IN, teda $\mathcal{O}(n^2)$.

\subsection*{COMPUTE}

Zložením postupnosti operácií REDUCE na vrcholoch $1 \dots (n-2)$ a následne AUGMENT na vrcholoch v opačnom poradí získame výslednú maticu najkratších ciest pre pôvodný graf $G$:\\
\ \\
\begin{algorithmic}[1]
    \Function{COMPUTE}{$G[1 \dots n][1 \dots n]$}
        \State{$\mathcal{G} \gets \mathbf{copy}\ G$}
        \For{$i \gets 1 \dots n - 2$}
            \State{$\mathcal{G} \gets REDUCE(G, \mathcal{G}, n)$}
        \EndFor
        \For{$i \gets (n - 2) \dots 1$}
            \State{$\mathcal{G} \gets AUGMENT(G, \mathcal{G}, n)$}
        \EndFor
        \State \Return{$\mathcal{G}$}
    \EndFunction
\end{algorithmic}
\ \\
Kedže sa vo funkcii nachádza cyklus v ktorom sa opakovane volá funkcia REDUCE/AUGMENT so zložitosťou $\mathcal{O}(n^2)$, výsledná funkcia COMPUTE je $\mathcal{O}(n^3)$.

\subsection*{NEGATIVE CYCLE DETECTION}

Negatívny $k$-cyklus je cesta $u \leadsto u$ v tvare $u_1 \to \dots \to u_k \to u_1$ ktorej celkové ohodnotenie je záporné. $G$ obsahuje negatívny cyklus $\iff$ ak po vykonaní REDUCE na vrcholoch $u_2, \dots, u_{k-1}$ redukovaný graf obsahuje negatívny $2$-cyklus v tvare $u_1 \to u_k \to u_1$.\\
\ \\
Táto situácia je jednoducho detekovatelná pri opätovnom vykonaní REDUCE na $u_1$ alebo $u_k$. Kedže funkcia COMPUTE vykoná redukciu $(n-2)$-krát, jediný špeciálny prípad nastane ak $u_1$ a $u_k$ su práve posledné dva vrcholy na ktoré sa už REDUCE neaplikuje. Nasledujú výpisy obsahujúce zmeny:\\
\ \\
\begin{algorithmic}[1]
    \Function{REDUCE\_NCD}{$G[1 \dots n][1 \dots n], node$}
        \State{$\vdots$}
        \If{$G[i][j] = \perp \vee \ G[i][j] > cost$}
            \State{$G[i][j] \gets cost$}
            \If{$G[j][i] \neq \perp \wedge \ G[i][j] + G[j][i] < 0$}
                \State \Return{$\perp$}
            \EndIf    
        \EndIf
        \State{$\vdots$}
    \EndFunction
\end{algorithmic}
\ \\
\begin{algorithmic}[1]
    \Function{COMPUTE\_NCD}{$G[1 \dots n][1 \dots n]$}
        \State{$\mathcal{G} \gets \mathbf{copy}\ G$}
        \For{$i \gets 1 \dots (n - 2)$}
            \State{$\mathcal{G} \gets REDUCE\_NCD(G, \mathcal{G}, n)$}
            \If{$\mathcal{G} = \perp$}
                \State \Return{$\perp$}
            \EndIf
        \EndFor
        \If{$G[n - 1][n] + G[n][n - 1] < 0$}
            \State \Return{$\perp$}
        \EndIf
        \State{$\vdots$}
    \EndFunction
\end{algorithmic}
\ \\
Ak matica najkratších ciest neexistuje, algoritmus vráti $\perp$.

\pagebreak

%----------------------------------------------------------------------------------------
%	Príklad 4
%----------------------------------------------------------------------------------------

\section*{IV003 2014, Sada 3, Príklad 4}
\subsection*{Jakub Senko (373902), Štefan Uherčík (374375)}



\end{document}
