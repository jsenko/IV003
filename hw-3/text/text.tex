%%%%%%%%%%%%%%%%%%%%%%%%%%%%%%%%%%%%%%%%%
% Short Sectioned Assignment
% LaTeX Template
% Version 1.0 (5/5/12)
%
% This template has been downloaded from:
% http://www.LaTeXTemplates.com
%
% Original author:
% Frits Wenneker (http://www.howtotex.com)
%
% License:
% CC BY-NC-SA 3.0 (http://creativecommons.org/licenses/by-nc-sa/3.0/)
%
%%%%%%%%%%%%%%%%%%%%%%%%%%%%%%%%%%%%%%%%%

%----------------------------------------------------------------------------------------
%	PACKAGES AND OTHER DOCUMENT CONFIGURATIONS
%----------------------------------------------------------------------------------------

\documentclass[paper=a4, fontsize=11pt]{scrartcl} % A4 paper and 11pt font size

\usepackage[T1]{fontenc} % Use 8-bit encoding that has 256 glyphs
% \usepackage{fourier} % Use the Adobe Utopia font for the document - comment this line to return to the LaTeX default
\usepackage[slovak]{babel} % Slovak language/hyphenation
\usepackage[utf8]{inputenc}
% \usepackage[IL2]{fontenc}
\usepackage{amsmath,amsfonts,amsthm} % Math packages
\usepackage{algpseudocode}
\usepackage{lipsum} % Used for inserting dummy 'Lorem ipsum' text into the template

\usepackage{sectsty} % Allows customizing section commands
\allsectionsfont{\centering \normalfont\scshape} % Make all sections centered, the default font and small caps

\usepackage{fancyhdr} % Custom headers and footers
\pagestyle{fancyplain} % Makes all pages in the document conform to the custom headers and footers
\fancyhead{} % No page header - if you want one, create it in the same way as the footers below
\fancyfoot[L]{} % Empty left footer
\fancyfoot[C]{} % Empty center footer
\fancyfoot[R]{\thepage} % Page numbering for right footer
\renewcommand{\headrulewidth}{0pt} % Remove header underlines
\renewcommand{\footrulewidth}{0pt} % Remove footer underlines
\setlength{\headheight}{13.6pt} % Customize the height of the header

\numberwithin{equation}{section} % Number equations within sections (i.e. 1.1, 1.2, 2.1, 2.2 instead of 1, 2, 3, 4)
\numberwithin{figure}{section} % Number figures within sections (i.e. 1.1, 1.2, 2.1, 2.2 instead of 1, 2, 3, 4)
\numberwithin{table}{section} % Number tables within sections (i.e. 1.1, 1.2, 2.1, 2.2 instead of 1, 2, 3, 4)

\setlength\parindent{0pt} % Removes all indentation from paragraphs - comment this line for an assignment with lots of text

%----------------------------------------------------------------------------------------
%	TITLE SECTION
%----------------------------------------------------------------------------------------

\newcommand{\horrule}[1]{\rule{\linewidth}{#1}} % Create horizontal rule command with 1 argument of height

\title{	
\normalfont \normalsize 
\textsc{IV003, Fakulta Informatiky, Masarykova Univerzita} \\ [25pt] % Your university, school and/or department name(s)
\horrule{0.5pt} \\[0.4cm] % Thin top horizontal rule
\huge Homework 3 \\ % The assignment title
\horrule{2pt} \\[0.5cm] % Thick bottom horizontal rule
}

\author{Jakub Senko, Štefan Uherčík} % Your name

\date{\normalsize\today} % Today's date or a custom date

\begin{document}

%\maketitle  Print the title

%----------------------------------------------------------------------------------------
%	Príklad 1
%----------------------------------------------------------------------------------------

\section*{IV003 2014, Sada 3, Príklad 1}
\subsection*{Jakub Senko (373902), Štefan Uherčík (374375)}

\pagebreak

%----------------------------------------------------------------------------------------
%	Príklad 2
%----------------------------------------------------------------------------------------

\section*{IV003 2014, Sada 3, Príklad 2}
\subsection*{Jakub Senko (373902), Štefan Uherčík (374375)}

\pagebreak

%----------------------------------------------------------------------------------------
%	Príklad 3
%----------------------------------------------------------------------------------------

\section*{IV003 2014, Sada 3, Príklad 3}
\subsection*{Jakub Senko (373902), Štefan Uherčík (374375)}

Vstupom algoritmu je matica reprezentujuca zadany graf, kde cislo $G[i][j]$ je ohodnotenie hrany $i \to j$. Pre neexistujucu hranu zavedieme specialnu hodnotu $\perp$.

\subsection*{Reduce}

Navrhneme funkciu REDUCE, ktora odstrani vrchol $node$ tak, aby sa nezmenila dlzka najkratsich ciet medzi ostatnymi vrcholmi.\\
\ \\
V pripade, ak vrcholom $node$ vedie najkratsia cesta $u \leadsto i \to node \to j \leadsto v$, je potrebne nahradit podcestu $i \to node \to j$ novou hranou $i \to j$ tak, aby $G[i][j] = G[i][node] + G[node][j]$. Ak hrana $i \to j$ uz existuje, staci ponechat hranu s nizsim ohodnotenim pretoze najkratsia cesta bude nutne obsahovat kratsiu hranu.\\
\ \\
Tuto operaciu je nutne vykonat pre vsetky mozne podcesty prechadzajuce vrcholom $node$. Algoritmus teda obsahuje dva vnorene cykly. Vonkajsi cyklus iteruje cez vsetky vstupne hrany $i \to node$ a vnutorny cyklus cez vystupne hrany $node \to j$. Vo vnutornom cykle sa vytvori nova hrana alebo nahradi, ak je ohodnotenie nizsie ako existujuce. Po skonceni funkcie su vsetky ohodnotenia pre dany vrchol nastavene na $\perp$ co reprezentuje zrusenie vrcholu.\\
\ \\
\begin{algorithmic}[1]
    \Function{REDUCE}{$G[1 \dots n][1 \dots n], node$}
        \For{$i \gets 1 \dots n$}
            \If{$G[i][node] \neq \perp \wedge \ i \neq node$}
                \For{$j \gets 1 \dots n$}
                   \If{$G[node][j] \neq \perp \wedge \ j \neq node$}
                       \State{$cost \gets G[i][node] + G[node][j]$}
                       \If{$G[i][j] = \perp \vee \ G[i][j] > cost$}
                           \State{$G[i][j] \gets cost$}
                       \EndIf
                   \EndIf
                \EndFor                
                \State{$G[i][node] \gets \perp$}
            \EndIf            
        \EndFor
       \For{$j \gets 1 \dots n$}
           \State{$G[node][j] \gets \perp$}
       \EndFor
        \State \Return{$G$}
    \EndFunction
\end{algorithmic}

Zlozitost je $\mathcal{O}(n^2)$.

\subsection*{AUGMENT}

Predpokladame ze matica grafu najkratsich ciest $\mathcal{G}$ ma rovnake rozmery ako matica $G$ reprezentujuca povodny graf. Vsetky ohodnotenia hran v $\mathcal{G}$ su rovne dlzke najkratsej cesty medzi danymi vrcholmi. Neexistujuci vrchol je znovu reprezentovany tak, ze ohodnotenia vsetkych pripojenych hran su $\perp$.\\
\ \\
Vstupom nasledovneho algoritmu je povodny graf $G$, $\mathcal{G}$ a vrchol $node \in V_{G} \setminus V_{\mathcal{G}}$. Vysledkom algoritmu je matica najkratsich ciest obsahujuca $node$.\\
\ \\
Funkcia je oddelena pre vstupne a vystupne hrany. Pre vystupne hrany:
Nech $u \to v$ je hrana v $\mathcal{G}$ Pre kazdu hranu $node \to u$ v povodnom grafe $G$ pridame do $\mathcal{G}$ hranu $node \to v$ s ohodnotenim $G[node][u] +\mathcal{G}[u][v]$. Kedze vrchol u moze by rozny tak tato hrana uz moze existovat. Znovu pouzijeme nizsie ohodnotenie.\\
\ \\
\begin{algorithmic}[1]
    \Function{AUGMENT\_OUT}{$G[1 \dots n][1 \dots n], \mathcal{G}[1 \dots n][1 \dots n], node$}
        \State{$\mathcal{G}[node][node] \gets 0$}
        \For{$u \gets 1 \dots n$}
            \If{$G[node][u] \neq \perp \wedge \ u \neq node$}
                \For{$v \gets 1 \dots n$}
                    \If{$\mathcal{G}[u][v] \neq \perp \wedge \ v \neq node$}
                        \State{$cost \gets G[node][u] + \mathcal{G}[u][v]$}
                        \If{$\mathcal{G}[node][v] = \perp \vee \ \mathcal{G}[node][v] > cost$}
                            \State{$\mathcal{G}[node][v] \gets cost$}
                        \EndIf
                    \EndIf
                \EndFor
            \EndIf
        \EndFor
        \State \Return{$\mathcal{G}$}
    \EndFunction
\end{algorithmic}
\ \\
Podobne vyzera funkcia AUGMENT\_IN, jedinym rozdielom je zmena smeru hran. Pre kratkost vypis tejto funkcie vynecham. Vysledna funkcia AUGMENT je nasledovna\\
\ \\
\begin{algorithmic}[1]
    \Function{AUGMENT}{$G[1 \dots n][1 \dots n], \mathcal{G}[1 \dots n][1 \dots n], node$}
        \State{$\mathcal{G} \gets AUGMENT\_IN(G, \mathcal{G}, node)$}
        \State{$\mathcal{G} \gets AUGMENT\_OUT(G, \mathcal{G}, node)$}
        \State \Return{$\mathcal{G}$}
    \EndFunction
\end{algorithmic}

\subsection*{COMPUTE}

Zlozenim postupnosti operacii REDUCE na vrcholoch $1 \dots (n-2)$ a AUGMENT v opacnom poradi ziskame vyslednu maticu najkratsich ciest pre povodny graf $G$.

Funkcia vyzera nasledovne:\\
\ \\
\begin{algorithmic}[1]
    \Function{COMPUTE}{$G[1 \dots n][1 \dots n]$}
        \State{$\mathcal{G} \gets \mathbf{copy}\ G$}
        \For{$i \gets 1 \dots n - 2$}
            \State{$\mathcal{G} \gets REDUCE(G, \mathcal{G}, n)$}
        \EndFor
        \For{$i \gets n - 2 \dots 1$}
            \State{$\mathcal{G} \gets AUGMENT(G, \mathcal{G}, n)$}
        \EndFor
        \State \Return{$\mathcal{G}$}
    \EndFunction
\end{algorithmic}

\pagebreak

%----------------------------------------------------------------------------------------
%	Príklad 4
%----------------------------------------------------------------------------------------

\section*{IV003 2014, Sada 3, Príklad 4}
\subsection*{Jakub Senko (373902), Štefan Uherčík (374375)}



\end{document}
