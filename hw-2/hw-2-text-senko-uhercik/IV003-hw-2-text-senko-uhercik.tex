%%%%%%%%%%%%%%%%%%%%%%%%%%%%%%%%%%%%%%%%%
% Short Sectioned Assignment
% LaTeX Template
% Version 1.0 (5/5/12)
%
% This template has been downloaded from:
% http://www.LaTeXTemplates.com
%
% Original author:
% Frits Wenneker (http://www.howtotex.com)
%
% License:
% CC BY-NC-SA 3.0 (http://creativecommons.org/licenses/by-nc-sa/3.0/)
%
%%%%%%%%%%%%%%%%%%%%%%%%%%%%%%%%%%%%%%%%%

%----------------------------------------------------------------------------------------
%	PACKAGES AND OTHER DOCUMENT CONFIGURATIONS
%----------------------------------------------------------------------------------------

\documentclass[paper=a4, fontsize=11pt]{scrartcl} % A4 paper and 11pt font size

\usepackage[T1]{fontenc} % Use 8-bit encoding that has 256 glyphs
% \usepackage{fourier} % Use the Adobe Utopia font for the document - comment this line to return to the LaTeX default
\usepackage[slovak]{babel} % Slovak language/hyphenation
\usepackage[utf8]{inputenc}
% \usepackage[IL2]{fontenc}
\usepackage{amsmath,amsfonts,amsthm} % Math packages
\usepackage{algpseudocode}
\usepackage{lipsum} % Used for inserting dummy 'Lorem ipsum' text into the template

\usepackage{sectsty} % Allows customizing section commands
\allsectionsfont{\centering \normalfont\scshape} % Make all sections centered, the default font and small caps

\usepackage{fancyhdr} % Custom headers and footers
\pagestyle{fancyplain} % Makes all pages in the document conform to the custom headers and footers
\fancyhead{} % No page header - if you want one, create it in the same way as the footers below
\fancyfoot[L]{} % Empty left footer
\fancyfoot[C]{} % Empty center footer
\fancyfoot[R]{\thepage} % Page numbering for right footer
\renewcommand{\headrulewidth}{0pt} % Remove header underlines
\renewcommand{\footrulewidth}{0pt} % Remove footer underlines
\setlength{\headheight}{13.6pt} % Customize the height of the header

\numberwithin{equation}{section} % Number equations within sections (i.e. 1.1, 1.2, 2.1, 2.2 instead of 1, 2, 3, 4)
\numberwithin{figure}{section} % Number figures within sections (i.e. 1.1, 1.2, 2.1, 2.2 instead of 1, 2, 3, 4)
\numberwithin{table}{section} % Number tables within sections (i.e. 1.1, 1.2, 2.1, 2.2 instead of 1, 2, 3, 4)

\setlength\parindent{0pt} % Removes all indentation from paragraphs - comment this line for an assignment with lots of text

%----------------------------------------------------------------------------------------
%	TITLE SECTION
%----------------------------------------------------------------------------------------

\newcommand{\horrule}[1]{\rule{\linewidth}{#1}} % Create horizontal rule command with 1 argument of height

\title{	
\normalfont \normalsize 
\textsc{IV003, Fakulta Informatiky, Masarykova Univerzita} \\ [25pt] % Your university, school and/or department name(s)
\horrule{0.5pt} \\[0.4cm] % Thin top horizontal rule
\huge Homework 2 \\ % The assignment title
\horrule{2pt} \\[0.5cm] % Thick bottom horizontal rule
}

\author{Jakub Senko, Štefan Uherčík} % Your name

\date{\normalsize\today} % Today's date or a custom date

\begin{document}

%\maketitle  Print the title

%----------------------------------------------------------------------------------------
%	Príklad 1
%----------------------------------------------------------------------------------------

\section*{IV003 2014, Sada 2, Príklad 1}
\subsection*{Jakub Senko (373902), Štefan Uherčík (374375)}

Zaveďme všeobecnú reprezentáciu budov. Každá uvažovaná budova sa dá reprezentovať ako množina dvojíc $(x_k, h_k)$, určujúcich výšku budovy $h$ na súradnici $x$. Zápis sa dá zjednodušiť usporiadaním bodov vzostupne podľa $x$. Stačí uvažovať len tie dvojice, ktoré označujú miesto, v ktorom nastáva zmena výšky budovy. Tento zápis je ekvivalentný so zápisom použitým v zadaní

\begin{equation}
    \begin{aligned}
        (1, \boldsymbol{5}, 5) \sim ((1, 5), (5, 0)) \\
    \end{aligned}
\end{equation}

ide len o vnútornú reprezentáciu za účelom zjednodušenia algoritmu. \\
\\

\subsection*{Merge}

Uvažujme algoritmus MERGE, ktorý z reprezentácie dvoch budov vypočíta reprezentáciu ich siluety. \\
\\
Algoritmus využíva object BUILDING\_ITERATOR pomocou ktorého je možné postupne prechádzat reprezentáciou danej budovy. Obsahuje tri  metódy. \\
NEXT\_COORDINATE\_EXISTS a NEXT\_COORDINATE\_POSITION sú triviálne a neposúvajú pozíciu iterátora. Tretia metóda, GET\_HEIGHT($x$) vráti výšku budovy na zadanej súradnici. Táto metóda spôsobí dostatočný posun iterátora v prípade, že zadaná pozícia je väčšia alebo rovná ako NEXT\_COORDINATE\_POSITION\footnote{Ak iterátor prešiel všetky súradnice, NEXT\_COORDINATE\_POSITION vráti $\infty$ a GET\_HEIGHT je 0.}. Kedže iterátor je jednorázový, túto metódu nie je možné zavolať s argumentom menším ako v predchádzajúcom volaní. Iterátor si jednoducho pamätá poslednú výšku. \\
\\
Samotný MERGE pracuje s dvoma iterátormi, pre každú budovu jeden a výstup postupne ukladá do samostatného zoznamu. Základom je {\em while} smyčka, ktorá sa vykoná ak aspoň pre jeden s iterátorov platí NEXT\_COORDINATE\_EXISTS.
Algoritmus potom vybere menšie $x$ z NEXT\_COORDINATE\_POSITION a zavolá metódu GET\_HEIGHT na oboch iterátoroch. Následne vybere väčšiu z výšok, $h$ a zavolá funkciu TRY\_ADD, ktorá jednoducho vloží novú súradnicu $(x, h)$ do výsledného zoznamu v prípade, že sa výška siluety zmenila\footnote{Čo nemusí nastať}. \\
\\
Tento algoritmus funguje pre ľubovolné reprezentácie s dĺžkou $n_1, n_2$ v čase $\mathcal{O}(n_1 + n_2)$, čo je $\mathcal{O}(n)$ pre budovy s rovnako veľkou reprezentáciou. Zdôvodnenie je jednoduché - využíva jednosmerný iterátor na jedno použitie pre  každú reprezentáciu - a teda každú súradnicu spracuje práve raz. Algoritmus je konečný pretože pri každom priechode cyklom metóda GET\_HEIGHT posunie aspoň jeden z iterátorov. \\
\\

\subsection*{Rozdeľ a panuj}

Výslednú siluetu dosiahneme aplikovaním funkcie MERGE na vhodné podproblémy. Toto delenie funguje rovnako ako pri algoritme {\em merge sort}. Funkcia COMPUTE\_SILHOUETTE zoberie ako argument množinu reprezentácii budov. Ak táto množina obsahuje jednu budovu, tak ju vráti. Ak dve budovy, zavolá na nich MERGE a vráti výsledok. Ak viac, rozdelí množinu na dve rovnaké (s rozdielom jednej budovy v prípade nepárneho počtu) množiny, rekurzívne sa na oboch zavolá a výsledok znovu spojí pomocou MERGE a vráti. Týmto spôsobom funkcia COMPUTE\_SILHOUETTE vždy vráti merge všetkých svojich argumentov (nezávisí na poradí). \\
\\
Zložitosť závisí na počte MERGE operácii a veľkosti ich vstupu. Na každej úrovni rekurzie je suma veľkosti všetkých reprezentácií rovnaká ($n$ dĺžky 2 na začiatku vs dve dlhé $n$ na konci, kde $n$ je počet budov) a počet úrovní je $\log_2 n$. Výsledná zložitosť je teda $\mathcal{O}(n \log n)$

\pagebreak

%----------------------------------------------------------------------------------------
%	Príklad 2
%----------------------------------------------------------------------------------------

\section*{IV003 2014, Sada 2, Príklad 2}
\subsection*{Jakub Senko (373902), Štefan Uherčík (374375)}

Slovník pojmov: \ \\
slovo - reťazec, pre ktorý vráti funkcia dict true \ \\
veta - postupnosť zreťazených slov, medzi jednotlivými slovami sa nenachádzajú žiadne iné znaky \ \\

Algoritmus overí všetky možné rozdelenia vstupného reťazca na 2 časti, prefix a suffix.
Ak je daný reťazec veta, existuje aspoň jedno rozdelenie také, že suffix je slovo a prefix je znovu veta.
Tento test sa vykoná v tele cyklu, kde sa rekurzívne použije IS\_SENTENCE na danom prefixe vstupného slova.
Premenná $result$ je true práve vtedy ak je aspoň jeden z týchto testov je true. \ \\

\begin{algorithmic}[1]
    \Function{is\_sentence}{$w[1 \dots n]$}
        \State result = DICT($w[1 \dots n$]);
        \For{$i = 1 \dots n$}
        	\State item = IS\_SENTENCE($w[1 \dots i]$) $\wedge$ DICT($w[i + 1 \dots n]$);
		\State result = result $\vee$ item;
        \EndFor
        \State \Return result;
    \EndFunction
\end{algorithmic}
\ \\

Pri každom z rekurzívnych volaní je argument funkcie IS\_SENTENCE {\em vždy prefixom} argumentu volajúcej funkcie.
Pri tomto rekurzívnom algoritme môže existovať v strome rekurzie viacero ciest k volaniu IS\_SENTENCE s rovnakým argumentom, avšak každý reťazec má konečné množstvo prefixov, konkrétne $n - 1$.
Preto stačí vykonať iba lineárne množstvo volaní.
Tieto volania sa dajú {\em usporiadať} tak, že sa funkcia postupné volá pre rastúci prefix. 
Táto myšlienka je základom nasledujúcej dynamickej varianty funckie IS\_SENTENCE.\\

Predstavme si nasledovný prípad: \ \\
Na vstup dostane algoritmus reťazec o dĺžke $n$. \ \\
Pri overovaní jednotlivých rozdelení nájde slovo o dĺžke $a$ (a zároveň sa rekurzívne zanorí na prefixe o dĺžke $(n-a)$) a ďalej pokračuje v overovaní ďalších rozdelení (s prefixami dĺžky $(n-a+1), (n-a+2), \dots$). \ \\

Algoritmus sa rekurzívne zavolá na reťazci o dĺžke $n-a$.
Pri týchto volaniach však overuje prefixy s dĺžkami $1 \dots n-a$. Tieto prefixy však overoval aj predošlý priechod algoritmu. \ \\

Môžeme povedať, že volanie funkcie IS\_SENTENCE aplikované na reťazci dĺžky $n$ je závislé na všetkých možných volaniach funkcie IS\_SENTENCE aplikovaných na prefixoch tohto reťazca. \ \\

Z tohoto dôvodu je výhodné, ak vypočítame IS\_SENTENCE na prefixoch pôvodného reťazca a výsledky týchto volaní si uložíme do rovnomenného asociatívneho poľa.
Kľúč tohto poľa bude tvorený prefixom pôvodného reťazca, hodnota bude typu boolean. \ \\


\begin{algorithmic}[1]
    \Function{verify\_sentence}{$w[1 \dots n]$}
    \For{$i = 1 \dots n$}
	\State result = DICT($w[1 \dots i]$);
        \For{$j = 1 \dots i - 1$}
                \State item = IS\_SENTENCE($w[1 \dots j]$) $\wedge$ DICT($w[j + 1 \dots i - 1]$);
		\State result = result $\vee$ item;
        \EndFor
	\State IS\_SENTENCE($w[1 \dots i]$) = result;
	\EndFor
	\State \Return  IS\_SENTENCE($w[1 \dots n]$);
    \EndFunction	
\end{algorithmic}
\ \\
 
\section*{Korektnosť:}

\subsection*{Konvergencia:}
Prvý cyklus sa vykoná pre každý prefix slova $w$.
V cykle sa hodnota premennej $i$ reprezentujucej dlžku prefixu pri každom priechode zvýši o 1 a iterovanie skončí, keď premenná i dosiahne hodnotu $n$. \ \\

Vnorený cyklus sa vykoná pre každý prefix tohto prefixu. \ \\

Jediné miesta, u ktorých hrozí, že algoritmus nezastaví, sú volania cyklov. \ \\

V cykle s iterujúcou premennou $j$ sa zaručene pri každom priechode zvýši hodnota premennej $j$ a iterovanie skončí, keď premenná dosiahne hodnotu $i-1$. Jedná sa o for-from-to cyklus, takže v prípade že je iniciálna hodnota $j$ väčšia ako $i-1$ tak sa cyklus nevykoná vôbec.

\subsection*{Parciálna korektnosť:}
Dôkaz, že algoritmus vráti true pri validnej postupnosti slov pomocou matematickej indukcie: \ \\

\textbf{Bázový krok:} \ \\
 $S_0 = w_0$ je veta ktorá sa skladá z jedného slova $w_0$ ($w_0$ sa nachádza v slovníku). Pre takúto vetu vráti algoritmus true. Dôvod: položky result pre prvok $IS\_SENTENCE(w_0)$ budú obsahovať položku $dict(w_0)$, ktorá sa vyhodnotí na true. Na položky je aplikovaný operátor logiký súčet a pretože obsahujú minimálne jednu položku z hodnotou true, algoritmus vráti true. \ \\
	
\textbf{Indukčný krok:} \ \\
Predpokladáme, že pre vetu $s_1$ zloženú z $k$ slov: $s_1 = w_1.w_2...w_k$ vráti korektnú odpoveď true. \ \\

Predpokldáme, že pre vetu $s_2$ zloženú $s_2 = s_1.w_{k +1}$ algoritmus taktiež vráti true. \ \\

Jeden z prefixov vety $s_2$ musí byť reťazec zložený zo slov $w_1...w_k$.
Keďže algoritmus prechádza všetky prefixy, nastane situácia, jedna z položiek results pre prvok $s_2$ bude IS\_SENTENCE($s_1$) $\wedge$ DICT($w_{k+1}$).
	
	
	



 
\section*{Zložitosť}
Cyklus s iterujúcou premennou $i$ bude vykonaný $n$-krát. \ \\

Vnorený cyklus s iterujúcou premennou $j$ bude vykonaný $i-1$ krát, a plaží, že $i-1 < n$. Všetky operácie použité v ňom majú konštantnú zložitosť.
Celková zložitosť bude teda: $\mathcal{O}(n^2)$.

\pagebreak

%----------------------------------------------------------------------------------------
%	Príklad 3
%----------------------------------------------------------------------------------------

\section*{IV003 2014, Sada 2, Príklad 3}
\subsection*{Jakub Senko (373902), Štefan Uherčík (374375)}

Vstupom algoritmu bude:
pole pravdepodobností: $C[{p}_1,..,{p}_n]$\\
$k$ - počet padnutých orlov\\
PVD - pravdepodobnostná funkcia\ \\

Problém je možné definovať nasledovne: \\
Vypočítať pravdepodobnosť, že v poli padne $k$ orlov z $n$ mincí.
Táto pravdepodobnosť je ekvivaletná súčtu pravdepodobností nasledovných prípadov: \ \\

1.) Pravdepodobnosť prípadu, že posledná minca bude orol \ \\
Táto pravdepodobnosť je ekvivalentná súčinu čísla ${p}_n$ a pravdepodobnosti, že medzi prvými $n-1$ mincami bude $k-1$ orlov. \ \\

2.) pravdepodobnosť prípadu, že posledná minca nebude orol \ \\
táto pravdepodobnosť je ekvivalentná súčinu čísla $(1-{p}_n)$ a pravdepodobnosti, že medzi prvými $n-1$ mincami bude $k$ orlov. \ \\

Tento poznatok nám umožňuje definovať jednoduchý rekurzívny algoritmus (v ktorom zároveň ošetrujeme krajné prípady $n=k$ a $n=0$) \ \\

\begin{algorithmic}[1]
    \Function{PVD}{$C[{p}_1,..,{p}_n],k$}
        \If{$k=0$}
            \State \Return PVD($C[{p}_1,..,{p}_{n-1}],0)*(1-{p}_n$);
        \EndIf
        \If{$k=n$}
            \State \Return PVD($C[{p}_1,..,{p}_{n-1}],k-1)*p(n)$;
        \EndIf
        \State \Return PVD($C[{p}_1,..,{p}_{n-1}],k-1)*{p}_{n} + PVD(C[{p}_1,..,{p}_{n-1}],k)*(1-{p}_{n})$;
    \EndFunction
\end{algorithmic}
\ \\

Ak rozpíšeme vetvenie algoritmu vykonávanie bude vyzerať približne nasledovne: \ \\ \ \\
$PVD(C[{p}_1,..,{p}_n],k) = PVD(C[{p}_1,..,{p}_{n-1}],k-1)*{p}_n + PVD(C[{p}_1,..,{p}_{n-1}],k)*(1-{p}_n)$ \ \\

$PVD(C[{p}_1,..,{p}_{n-1}],k) = $ \ \\
	$         PVD(C[{p}_1,..,{p}_{n-2}],k-1)*{p}_{n-1} + PVD(C{p}_1,..,{p}_{n-2}],k)*(1-{p}_{n-1}))$ \ \\

$PVD(C[{p}_1,..,{p}_{n-1}],k-1) = $ \\ 
	$         PVD(C[{p}_1,..,{p}_{n-2}],k-2)*{p}_{n-1} + PVD(C[{p}_1,..,{p}_{n-2}],k-1)*(1-{p}_{n-1})$
...

\ \\ 

Z predchádzajúceho zápisu volaní funkcií je možné vidieť, že $PVD(C[{p}_1,..,{p}_{n-1}],k-1) $ sa zavolá 2 krát na druhej úrovni rekurzívneho stromu.
Využijeme techniku dynamického programovania, aby sme sa vyhli opakovanému volaniu funkcie PVD na rovnakých parametroch.
Z algoritmu je zreteľné, že volanie funkcie PVD, ktorá berie ako parameter pole o dĺžke a, je závislá výlučne na volaniach funkcií PVD, ktoré berú ako parameter pole o dĺžke a-1. \ \\

Z tohoto dôvodu je výhodné, ak vypočítame najprv. funkcie s parametrami $PVD(C[{p}_{1})],0)$, $PVD(C[{p}_{1}],1)$, $PVD(C[p_1,p_2)],0)$,... \ \\
 

Pre tento účel vytvoríme asociatívne pole s názvom PVD, v ktorom kľúče budú mať tvar:
(C[p(1),..,p(n)],k)
a hodnoty budú obsahovať napočítanú pravdepodobnosť.
Na naplnenie tohto poľa vytvoríme jednoduchú nerekurzívnu funkciu:

Ich výsledky si budem ukladať do asociatívneho poľa a postupným volaním sa dopracujem k hodnote PVD(C[p(1),..,p(n)],k), ktorá je výsledkom celého problému. \ \\


\begin{algorithmic}[1]
	\Function{$COUNT\_PVD$}{$C[p_1,..,p_n],k$}
		\State $PVD([p_1],0) = p_1$;
		\State $PVD([p_1],1) = (1-p_1)$;

		\For{$i = 1 .. n$}
			\State bottom = max(0,k-(n-i));
			\State up = min(i,k);
			\For{j = bottom .. up}
				\If{j==0}
					\State $PVD(C[p_1,..,p_{i}],j) = PVD(C[p_1,..,p_{i-1}],0)*(1-p_i);$
				\Else \If{k==n}
					\State $PVD(C[p_1,..,p_{i}],j) = PVD(C[p_1,..,p_{i-1}],j-1)*p_i;$
				\Else 
					\State $PVD(C[p_1,..,p_{i}],j) = $
					\State $PVD(C[p_1,..,p_{i-1}],j-1)*p_i + PVD(C[p_1,..,p_{i-1}],j)*(1-p_i);$
				\EndIf \EndIf
			\EndFor
		\EndFor
		\State \Return $PVD(C[p_{1},..,p_{n}],k);$
	\EndFunction
\end{algorithmic}
\ \\
\section*{Zložitosť}
Cyklus s iterujúcou premennou $i$ sa vykoná $n$-krát.
V ňom sa vnorený cyklus iterujúcou premennou $j$ vykoná vždy $(up - bottom)$ krát.
V každom cykle bude hodnota premennej $bottom$ minimálne 0 a hodnota premennej $up$ maximálne $k$, z čoho vyplýva, že počet týchto cyklov bude maximálne $k$.
Je zaručené, že $k < n$ a teda zložitosť celého algoritmu bude patriť do triedy $\mathcal{O}(n^2)$


\pagebreak

%----------------------------------------------------------------------------------------
%	Príklad 4
%----------------------------------------------------------------------------------------

\section*{IV003 2014, Sada 2, Príklad 4}
\subsection*{Jakub Senko (373902), Štefan Uherčík (374375)}

Cenu optimálneho rozdelenie áut do autosalónov je možné vypočítať pomocou jednoduchécho rekurzívneho algoritmu.
Maticu C chápeme ako n trojíc, pričom každá je v samostnatnom riadku. Číslo auta je zároveň jeho index riadku v matici cien.
Parameter n v algoritme bestPrice udáva počet áut ktoré ešte neboli priradené do autosalónu a kedže sa autá priradujú postupne, tak udáva zároveň aj číslo nasleujúceho nepriradeného auta.
Parameter freePlaces udáva počet neobsadených miest v jednotlivých autosalónoch (číslo na indexe i v tomto poli udáva počet volných miest v autosalóne i).
V každom rekurzívnom volaní priradíme auto do každého autosalónu az týchto priradení vrátime maximálnu cenu.
Rekurzia sa zastaví v prípade že sme priradili už všetky autá.

V nasledujúcich algoritmoch predpokladáme, že počet áut je delitelný počtom autsalónov. \ \\

\begin{algorithmic}[1]
\Function{best\_price}{n,freePlaces[]}
    \If{n == 0}
        \State \Return 0
    \EndIf

    \State values = [] of number

    \For{i = 1 .. freePlaces.size} 		
        \If{freePlaces[i] != 0}		
            \State freePlacesCopy = copy of freePlaces;			
            \State	freePlacesCopy[i] = freePlacesCopy[i] - 1;			
            \State values[i] = best\_price(n-1,freePlacesCopy) + C[n][i];
        \EndIf
    \EndFor

    \State \Return max(values);
\EndFunction
\end{algorithmic}

\subsection*{Technika dynamického programovania}

Každé volanie funkcie bestPrice na matici na k-tom riadku potrebuje  výsledky volaní funkcie bestPrice na riadkoch 1..k-1. Toto nám definuje závislosť  a teda možné usporiadanie volaní. \\

Vytvoríme štruktúru bestPriceData, ktorá  bude mať typ asociatívneho poľa, v ktorom: \\
Kľúče budú zodpovedať možným parametrom funkcie BEST\_PRICE - teda dvojica (n,freePlaces[])
hodnoty budú zodpovedať vypočítanej najvyššej možnej cene. \\


Aby sme zaznamenali všetky možné kombinácie parametrov n a freePlaces, vytvorili sme funkciu findSpecialPermutations. Táto funkcia vráti všetky možnosti ako môže vyzeraž pole freePlaces pre sum áut v prípade že do jedného autosalónu môžeme daž maximálne threshold áut. 
Aby sme ju popísali viac formálne, pre vstupné parametre sum a threshold hľadáme permutácie trojice čísel, pričom tieto čísla sú prirodzené alebo 0. Tieto čísla musia dávať súčet rovný parametru sum, ale zároveň žiadne z nich nemôže byť väčšie, ako hodnota threshold. \ \\

\begin{algorithmic}
\Function{findSpecialPermutations}{sum,threshold} 
	\If{sum = 0}
		\State \Return [0,0,0]
	\EndIf
	\State permutations = [] of [];
	\State bottom1 = max(0,sum - 2*threshold);
	\State upper1 = min(sum,threshold);
	
	\For{ i = bottom1 .. upper1 }
		\State permutation = [] of length 3
		\State permutation[1] = i;
		
		\State bottom2 = max(0,sum - i - threshold);
		\State upper2 = min(sum - i,threshold);
		
		\For{j=bottom2 .. upper2}
			\State permutation[2] = j;
			\State permutation[3] = sum - (i+j);
		\EndFor
		
		
		permutations.add(permutation);
	\EndFor
	\State \Return permutations;
\EndFunction
\end{algorithmic}
\ \\
Nasledovná nerekurzívna funkcia postupne priradí do štruktúry bestPriceData hodnoty najlepších možných cien pre danú konfiguráciu (počet priradených áut a počet predaných áut pre jednotlivé autosalóny (freePlaces)).
Vonkajší for cyklus s iterujúcou premennou i v každom priechode spocita maximalne ceny pre všetky distribucie (konkrétne priradenie do autosalónov) i áut.
Využívame hodnoty v bestPriceData vypočítané v predchádzajúcih priechodoch cyklom. \ \\


\begin{algorithmic}[1]
\Function{countBestPrice}{C}
    \State bestPriceData[ (0, [0,0,0]) ] = 0;
    \For{i = 1 .. C.rows}				
        \State permutations = findSpecialPermutations(i,C.rows/3);
	
		
        \For{permutation in permutations}		
            
            \State items = [] of number;
            
	    \For{ j = 1 .. permutation.size}
	    	
                \State modifiedPermutation = copy of permutation
		\If{modifiedPermutation[j]>0}
		    \State modifiedPermutation[j] = modifiedPermutation[j] - 1;
		    \State items.add(bestPriceData[(i-1,modifiedPermutation)] + C[i][j]);
                 \EndIf
            \EndFor			
		
            \State bestPriceData[(i,permutation)] = max(items);		
        \EndFor
    \EndFor
    \State \Return bestPriceData[C.rows,[C.rows/3,C.rows/3,C.rows/3]];
\EndFunction
\end{algorithmic}

\subsection*{Konvergencia}
Algoritmus môže nekonvergovať len pri vyhodnotení podmienky jedného z troch cyklov.
Cyklus s iterujúcou premennou j sa skončí pretože permutation.size je kladná konštanta.
Cyklus s iterujúcou premennou permutation skončí pretože funkcia findSpecialPermutations vráti konečný počet položiek.
Vonkajší cyklus skončí pretože C.rows je kladná konštanta.


\subsection*{Parciálna korektnosť}
Dokážeme matematickou indukciou podla počtu áut pre ktorý sme už vypočítali maximalnu cenu (vonkajší cyklus). \ \\

\textbf{Bázovy krok:} \ \\
Pre i = 0
Táto situácia je riešená pred cyklom naplnením pola bestPriceData hodnotou (0, [0,0,0]). \ \\ 

\textbf{Indukčný krok:} \ \\
Pre i = k
Predpokladáme že bestPriceData obsahuje správne vypočítané maximálne ceny pre všetky možné konfigurácie freePlaces pre počet áut k , označme ich currentFreePlaces
Vnútorný for cyklus s iterujúcou premennou permutation vypočíta maximálne ceny pre všetky možné konfigurácie freePlaces. Každá maximálna cena sa vypočíta ako maximum z už vypočítaných konfigurácii.
Pretože for cyklus prebehne pre všetky možné permutácie pre k áut, budú dostupné pre k+1 iteráciu cyklu.

\section{Zložitosť}
\textbf{funkcia findSpecialPermutations} \ \\
vo funkcii sa nachádzajú dva cykly:
počet vykonaní prvého cyklu je (upper1 - bottom1), rozdiel týchto premenných bude vždy menší ako počet áut.
Ekvivalentné tvrdenie môžeme spraviť pre druhý cyklus s počtom vykonaní (upper1 - bottom1).
Každá operácia použitá v cykloch má konštantnú zložitosť, preto je jej celková zložitosť $\mathcal{O}(n^2)$. \ \\

\textbf{funkcia countBestPrice} \ \\
vonkajší cyklus sa vykoná pre každé auto práve raz, teda C.row krát.

Zo zložistosti funkcie findSpecialPermutations môžeme povedať, že počet prvkov v poli permutations je maximálne kvadraticky závislý na premennej i.

Zložitosť operácií v cykle s iterujúcou premennou j je konštatný, vzhľadom na to, že hodnota permutation.size zodpovedá počtu autosalónov a ten je konštantný počas celého vykonávania programu.
Zložitosť cyklu s iterujúcou premennou permutation má preto zložitosť $\mathcal{O}(n^2)$ (kvôli počtu prvkov v poli permutations).

Predošlé tvrdenia môžeme zredukovať na tvrdenie, že v cykle s iterujúcou premennou i sa vždy vykonajú dve operácie so zložitosťou $\mathcal{O}(n^2)$.
Jeho zložitosť je teda $\mathcal{O}(n^3)$.


\section*{Rozšírenie algoritmu}

Pre krátkosť tento algoritmus počíta len výšku najlepšej ceny za akú je možné autá predať, ale je triviálne modfikovatelný aby si spolu s cenou pametal aj konkretne priradenie áut do autosalónov.
Namiesto ceny je stačí uložiť do bestPriceData dvojicu (cena, priradenie). Priradenie je pole o dlžke C.rows a hodnota na indexe i v tomto poli vyjadruje číslo autosalónu do ktorého bude auto priradené, prípadne 0 ak ešťe nie je priradené.

Nasledujúci algoritmus toto implementuje: \ \\




\begin{algorithmic}[1]
\Function{countBestPrice}{C}
    bestPriceData[(0,[0,0,0])] = 0;
    \For{i = 1 .. C.rows}		
        \State permutations = findSpecialPermutations(i,C.rows/3);
				
        \For{permutation in permutations}		
            \State items = [] of item (price -> number, distribution -> [] of number); 
            
	    \For{ j = 1 .. permutation.size}
                \State modifiedPermutation = copy of permutation
		\If{modifiedPermutation[j]>0}
		    \State modifiedPermutation[j] = modifiedPermutation[j] - 1;
		    \State	item it;
		    \State aPrice = bestPriceData[(i-1,modifiedPermutation)];
		    \State it.price = aPrice.price + C[i][j];
		    \State it.distribution = aPrice.distribution;
		    \State it.distribution[i] = j;
                    \State items.add(it);
                 \EndIf
            \EndFor			
		
            \State highestPriceIndex = index into items for item with highest price;
            \State bestPriceData[(i,permutation)] = items[highestPriceIndex];		
        \EndFor
    \EndFor
     \State \Return bestPriceData[C.rows,[C.rows/3,C.rows/3,C.rows/3]];
\EndFunction
\end{algorithmic}

\ \\

Algoritmus je rozšíriteľný pre väčší počet autosalónov, je však potrebné zmeniť implementáciu funkcie findSpecialPermutations.

\pagebreak

%----------------------------------------------------------------------------------------
%	Príklad 5
%----------------------------------------------------------------------------------------

\section*{IV003 2014, Sada 2, Príklad 5}
\subsection*{Jakub Senko (373902), Štefan Uherčík (374375)}

Hladový algoritmus nemusí nájsť optimálne riešenie pre druhý a tretí problém. \\
Protipríklad: \\
Dokument s dĺžkou riadku: 12 \\
Zoznam slov obsahuje slová s dĺžkami 5,4,4,12 \\

Algoritmus umiestni slová nasledovne: \\
5,4 \\
4 \\
12 \\

Druhý slovný problém: \\
Celková penalizácia bude $3^2 + 8^2 + 0 = 9 + 64 = 73$ \\

Optimálne riešenie je však: \\
5 \\
4,4 \\
12 \\
pri ktorom bude celková penalizácia $7^2 + 4^2 = 49 + 16 = 64$ \\

Tretí slovný problém: \\
Celková penalizácia bude: max(3,8,0) = 8 \\

Optimálne riešenie je však: \\
5 \\
4,4 \\
12 \\
pri ktorom bude celková penalizácia max(7,4,0) = 7 \\

\end{document}
