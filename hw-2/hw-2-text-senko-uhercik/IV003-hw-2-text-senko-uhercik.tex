%%%%%%%%%%%%%%%%%%%%%%%%%%%%%%%%%%%%%%%%%
% Short Sectioned Assignment
% LaTeX Template
% Version 1.0 (5/5/12)
%
% This template has been downloaded from:
% http://www.LaTeXTemplates.com
%
% Original author:
% Frits Wenneker (http://www.howtotex.com)
%
% License:
% CC BY-NC-SA 3.0 (http://creativecommons.org/licenses/by-nc-sa/3.0/)
%
%%%%%%%%%%%%%%%%%%%%%%%%%%%%%%%%%%%%%%%%%

%----------------------------------------------------------------------------------------
%	PACKAGES AND OTHER DOCUMENT CONFIGURATIONS
%----------------------------------------------------------------------------------------

\documentclass[paper=a4, fontsize=11pt]{scrartcl} % A4 paper and 11pt font size

\usepackage[T1]{fontenc} % Use 8-bit encoding that has 256 glyphs
% \usepackage{fourier} % Use the Adobe Utopia font for the document - comment this line to return to the LaTeX default
\usepackage[slovak]{babel} % Slovak language/hyphenation
\usepackage[utf8]{inputenc}
% \usepackage[IL2]{fontenc}
\usepackage{amsmath,amsfonts,amsthm} % Math packages
\usepackage{algpseudocode}
\usepackage{lipsum} % Used for inserting dummy 'Lorem ipsum' text into the template

\usepackage{sectsty} % Allows customizing section commands
\allsectionsfont{\centering \normalfont\scshape} % Make all sections centered, the default font and small caps

\usepackage{fancyhdr} % Custom headers and footers
\pagestyle{fancyplain} % Makes all pages in the document conform to the custom headers and footers
\fancyhead{} % No page header - if you want one, create it in the same way as the footers below
\fancyfoot[L]{} % Empty left footer
\fancyfoot[C]{} % Empty center footer
\fancyfoot[R]{\thepage} % Page numbering for right footer
\renewcommand{\headrulewidth}{0pt} % Remove header underlines
\renewcommand{\footrulewidth}{0pt} % Remove footer underlines
\setlength{\headheight}{13.6pt} % Customize the height of the header

\numberwithin{equation}{section} % Number equations within sections (i.e. 1.1, 1.2, 2.1, 2.2 instead of 1, 2, 3, 4)
\numberwithin{figure}{section} % Number figures within sections (i.e. 1.1, 1.2, 2.1, 2.2 instead of 1, 2, 3, 4)
\numberwithin{table}{section} % Number tables within sections (i.e. 1.1, 1.2, 2.1, 2.2 instead of 1, 2, 3, 4)

\setlength\parindent{0pt} % Removes all indentation from paragraphs - comment this line for an assignment with lots of text

%----------------------------------------------------------------------------------------
%	TITLE SECTION
%----------------------------------------------------------------------------------------

\newcommand{\horrule}[1]{\rule{\linewidth}{#1}} % Create horizontal rule command with 1 argument of height

\title{	
\normalfont \normalsize 
\textsc{IV003, Fakulta Informatiky, Masarykova Univerzita} \\ [25pt] % Your university, school and/or department name(s)
\horrule{0.5pt} \\[0.4cm] % Thin top horizontal rule
\huge Homework 2 \\ % The assignment title
\horrule{2pt} \\[0.5cm] % Thick bottom horizontal rule
}

\author{Jakub Senko, Štefan Uherčík} % Your name

\date{\normalsize\today} % Today's date or a custom date

\begin{document}

\maketitle % Print the title

%----------------------------------------------------------------------------------------
%	Príklad 1
%----------------------------------------------------------------------------------------

\section*{Príklad 1}

Zaveďme všeobecnú reprezentáciu budov. Každá uvažovaná budova sa dá reprezentovať ako množina dvojíc $(x_k, h_k)$, určujúcich výšku budovy $h$ na súradnici $x$. Zápis sa dá zjednodušiť usporiadaním bodov vzostupne podľa $x$. Stačí uvažovať len tie dvojice, ktoré označujú miesto, v ktorom nastáva zmena výšky budovy. Tento zápis je ekvivalentný so zápisom použitým v zadaní

\begin{equation}
    \begin{aligned}
        (1, \boldsymbol{5}, 5) \sim ((1, 5), (5, 0)) \\
    \end{aligned}
\end{equation}

ide len o vnútornú reprezentáciu za účelom zjednodušenia algoritmu. \\
\\

\subsection*{Merge}

Uvažujme algoritmus MERGE, ktorý z reprezentácie dvoch budov vypočíta reprezentáciu ich siluety. \\
\\
Algoritmus využíva object BUILDING\_ITERATOR pomocou ktorého je možné postupne prechádzat reprezentáciou danej budovy. Obsahuje tri  metódy. \\
NEXT\_COORDINATE\_EXISTS a NEXT\_COORDINATE\_POSITION sú triviálne a neposúvajú pozíciu iterátora. Tretia metóda, GET\_HEIGHT($x$) vráti výšku budovy na zadanej súradnici. Táto metóda spôsobí dostatočný posun iterátora v prípade, že zadaná pozícia je väčšia alebo rovná ako NEXT\_COORDINATE\_POSITION. Kedže iterátor je jednorázový, túto metódu je nie je možné zavolať s argumentom menším ako v predchádzajúcom volaní. Iterátor si jednoducho pamätá poslednú výšku. \\
\\
Samotný MERGE pracuje s dvoma iterátormi, pre každú budovu jeden a výstup postupne ukladá do samostatného zoznamu. Základom je {\em while} smyčka, ktorá sa vykoná ak aspoň pre jeden s iterátorov platí NEXT\_COORDINATE\_EXISTS.
Algoritmus potom vybere menšie $x$ z NEXT\_COORDINATE\_POSITION a zavolá metódu GET\_HEIGHT na oboch iterátoroch. Následne vybere väčšiu z výšok, $h$ a zavolá funkciu TRY\_ADD, ktorá jednoducho vloží novú súradnicu $(x, h)$ do výsledného zoznamu v prípade, že sa výška siluety zmenila (čo nemusí nastať). \\
\\
Tento algoritmus funguje pre ľubovolné reprezentácie s dĺžkou $n_1, n_2$ v čase $\mathcal{O}(n_1 + n_2)$, čo je $\mathcal{O}(n)$ pre budovy s rovnako veľkou reprezentáciou. Zdôvodnenie je jednoduché - využíva jednosmerný iterátor na jedno použitie pre  každú reprezentáciu - a teda každú súradnicu spracuje práve raz. Algoritmus je konečný pretože pri každom priechode cyklom metóda GET\_HEIGHT posunie aspoň jeden z iterátorov. \\
\\

\subsection*{Rozdeľ a panuj}

Výslednú siluetu dosiahneme aplikovaním funkcie MERGE na vhodné podproblémy. Toto delenie funguje rovnako ako pri algoritme {\em merge sort}. Funkcia COMPUTE\_SILHOUETTE zoberie ako argument množinu reprezentácii budov. Ak táto množina obsahuje jednu budovu, tak ju vráti. Ak dve budovy, zavolá na nich MERGE a vráti výsledok. Ak viac, rozdelí množinu na dve rovnaké (s rozdielom jednej budovy v prípade nepárneho počtu) množiny, rekurzívne sa na oboch zavolá a výsledok znovu spojí pomocou MERGE a vráti. Týmto spôsobom funkcia COMPUTE\_SILHOUETTE vždy vráti merge všetkých spojich argumentov (merge nezávisí na poradí). \\
\\
Zložitosť závisí na počte MERGE operácii a veľkosti ich vstupu. Na každej úrovni rekurzie je suma veľkosti všetkých reprezentácií rovnaká ($n$ dĺžky 2 na začiatku vs dve dlhé $n$ na konci, kde $n$ je počet budov) a počet úrovní je $\log_2 n$. Výsledná zložitosť je teda $\mathcal{O}(n \log n)$

\pagebreak

%----------------------------------------------------------------------------------------
%	Príklad 2
%----------------------------------------------------------------------------------------

\section*{Príklad 2}

\pagebreak

%----------------------------------------------------------------------------------------
%	Príklad 3
%----------------------------------------------------------------------------------------

\section*{Príklad 3}

\pagebreak

%----------------------------------------------------------------------------------------
%	Príklad 4
%----------------------------------------------------------------------------------------

\section*{Príklad 4}

\pagebreak

%----------------------------------------------------------------------------------------
%	Príklad 5
%----------------------------------------------------------------------------------------

\section*{Príklad 5}

\end{document}
